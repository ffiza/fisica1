\documentclass[a4paper, 12pt, fleqn]{article}

\usepackage[utf8]{inputenc}
\usepackage[spanish]{babel}
\usepackage{amsmath}
\usepackage{graphicx}
\usepackage{float}
\usepackage[dvipsnames]{xcolor}
\usepackage{tikz}
\usepackage{xfrac}
\usepackage{fullpage}
\usepackage{siunitx}
\usepackage{enumitem}
\usepackage{pgfplots}
\usepackage{pgfplotstable}
\usepackage[colorlinks, linktocpage=true]{hyperref}
\usepackage[explicit]{titlesec}

\usepackage{mystyle}

\newcommand{\version}{v2023.1.0}
\decimalpoint

\begin{document}

\begin{center}
    {\LARGE\sc F\'isica 1} \vspace{1ex} \\
    {\small PRIMER CUATRIMESTRE DE 2023} \vspace{2ex} \\
    {\large\sc Guía 0: Repaso}\footnote{\version} \vspace{2ex} \\
\end{center}

\section{}

Hallar el módulo del vector que tiene origen en $(20;-5;8)$ y extremo en $(-4;-3;2)$.

\section{}

Hallar las componentes cartesianas de los siguientes vectores:
    
\begin{center}
    \begin{tikzpicture}[scale=.8]
        \draw (-2, 2) node {(a)};
        \draw[line width=0.8, black, -latex] (-3,0) -- (3,0) node[below] {$x$};
        \draw[line width=0.8, black, -latex] (0,-3) -- (0,3) node[right] {$y$};
        \draw[line width=2, black, -latex] (0,0) -- (30:2) node[above] {$\left| {\bf A} \right| = 2$};
        \draw (20:1) node[right] {$\theta = \ang{30}$};
        \draw[thick, dashed] (1,0) arc (0:30:1);
    \end{tikzpicture}
    \hspace{.05cm}
    \begin{tikzpicture}[scale=.8]
        \draw (-2, 2) node {(b)};
        \draw[line width=0.8, black, -latex] (-3,0) -- (3,0) node[below] {$x$};
        \draw[line width=0.8, black, -latex] (0,-3) -- (0,3) node[right] {$y$};
        \draw[line width=2, black, -latex] (0,0) -- (240:3);
        \draw (200:3) node[right] {$\left| {\bf A} \right| = 3$};
        \draw (45:1.25) node[right] {$\theta = \ang{240}$};
        \draw[thick, dashed] (1,0) arc (0:240:1);
    \end{tikzpicture}
    \hspace{.05cm} 
    \begin{tikzpicture}[scale=.8]
        \draw (-2, 2) node {(c)};
        \draw[line width=0.8, black, -latex] (-3,0) -- (3,0) node[below] {$x$};
        \draw[line width=0.8, black, -latex] (0,-3) -- (0,3) node[right] {$y$};
        \draw[line width=2, black, -latex] (0,0) -- (270:1.5);
        \draw (205:3) node[right] {$\left| {\bf A} \right| = 1.5$};
        \draw (45:1.25) node[right] {$\theta = \ang{270}$};
        \draw[thick, dashed] (1,0) arc (0:270:1);
    \end{tikzpicture}
\end{center}

\section{}

Determinar el módulo y la dirección de los siguientes vectores y representarlos gráficamente.
\begin{enumerate}[label=(\alph*)]
    \item ${\bf A} = (3; 3) $
    \item ${\bf B} = (2; 0) $
    \item ${\bf C} = -4 \hat{x} - 3 \hat{y} $
    \item ${\bf D} = -5 \hat{x} $
\end{enumerate}

\section{}

¿Qué propiedades tienen los vectores ${\bf A}$ y ${\bf B}$ que cumplen las siguientes condiciones?
\begin{enumerate}[label=(\alph*)]
    \item ${\bf A} + {\bf B} = {\bf C} $ y $\left| {\bf A} \right| + \left| {\bf B} \right| = \left| {\bf C} \right| $
    \item ${\bf A} + {\bf B} = {\bf A} - {\bf B} $
    \item ${\bf A} + {\bf B} = {\bf C} $ y ${\bf A}^2 + {\bf B}^2 = {\bf C}^2 $
\end{enumerate}

\section{Producto escalar de dos vectores}
\label{ref:prod_escalar}

El {\it producto escalar} entre dos vectores se define como ${\bf A} \cdot {\bf B} = \left| {\bf A} \right| \left| {\bf B} \right| \cos \theta$, donde $\theta$ es el ángulo que forman los dos vectores. La base canónica de la terna derecha se define con los vectores $\hat{x} = (1;0;0)$, $\hat{y} = (0;1;0)$ y $\hat{z} = (0;0;1)$. Calcular $\hat{x} \cdot \hat{x}$, $\hat{x} \cdot \hat{y}$, $\hat{x} \cdot \hat{z}$, $\hat{y} \cdot \hat{y}$, $\hat{y} \cdot \hat{z}$, $\hat{z} \cdot \hat{z}$ y $\hat{y} \cdot \hat{x}$.

\section{}

Haciendo uso de la propiedad distributiva del producto escalar respecto de la suma, ${\bf C} \cdot \left( {\bf E} + {\bf F} \right) = {\bf C} \cdot {\bf E} + {\bf C} \cdot {\bf F}$, y de los resultados obtenidos en el ejercicio \ref{ref:prod_escalar}, demostrar que si ${\bf A} = A_x \hat{x} + A_y \hat{y} + A_z \hat{z} $ y ${\bf B} = B_x \hat{x} + B_y \hat{y} + B_z \hat{z} $, entonces $$ {\bf A} \cdot {\bf B} = A_x B_x + A_y B_y + A_z B_z. $$

\section{Teoremas del coseno y del seno}

\begin{enumerate}[label=(\alph*)]
    \item Utilizando el teorema de Pitágoras y la definición de las funciones trigonométricas, demostrar en el triángulo de la figura el teorema del coseno: $$ b^2 = c^2 + a^2 - 2ca \cos \beta. $$ Ayuda: considere los triángulos rectángulos ABD y ADC.
    
    \begin{center}
    \begin{tikzpicture}
        \draw[line width=0.3, black] (3,0) -- (3,2.5);
        \draw[line width=0.3, black] (3.2,0) -- (3.2,0.2) -- (2.8,0.2) -- (2.8,0);
        \draw[line width=1.5, black] (0,0) -- (5,0) -- (3,2.5) -- (0,0);
        
        \draw (0,0) node[below] {B};
        \draw (3,0) node[below] {D};
        \draw (5,0) node[below] {C};
        \draw (3,2.5) node[above] {A};
        
        \draw[line width=0.7, latex-latex] ([shift=(129:1)]5,0) arc (129:180:1);
        \draw[line width=0.7, latex-latex] (1,0) arc (0:39:1);
        \draw[line width=0.7, latex-latex] ([shift=(219:1)]3,2.5) arc (219:309:1);
        
        \draw (1.2,0.5) node {$\beta$};
        \draw (2.5,1.3) node {$\alpha$};
        \draw (3.8,0.5) node {$\gamma$};
        
        \draw (2.5,0) node[below] {$a$};
        \draw (1.5,1.25) node[above left] {$c$};
        \draw (4,1.25) node[above right] {$b$};
    \end{tikzpicture}
    \end{center}

    \item Utilizando la definición del seno, demostrar sobre los mismos triángulos que $$ \frac{b}{\sin \beta} = \frac{c}{\sin \gamma}. $$ Generalizar el resultado para demostrar el teorema del seno: $$ \frac{b}{\sin \beta} = \frac{c}{\sin \gamma} = \frac{a}{\sin \alpha}. $$
\end{enumerate}

\section{Producto vectorial de dos vectores}

Sean $\hat{x}$, $\hat{y}$ y $\hat{z}$ los versores de la terna mostrada en la figura. Usando la definición del producto vectorial, calcular:
    \begin{enumerate}[label=(\alph*)]
        \item $\hat{x} \times \hat{y}$
        \item $\hat{z} \times \hat{x}$
        \item $\hat{y} \times \hat{z}$
        \item $\hat{x} \times \hat{x}$
        \item $\hat{y} \times \hat{y}$
        \item $\hat{z} \times \hat{z}$
    \end{enumerate}

\begin{center}
    \begin{tikzpicture}
        \draw[line width=0.8, black, -latex] (0,0) -- (-30:2) node[below] {$y$};
        \draw[line width=0.8, black, -latex] (0,0) -- (210:2) node[below] {$x$};
        \draw[line width=0.8, black, -latex] (0,0) -- (90:2) node[right] {$z$};
        \draw[line width=2, black, -latex] (0,0) -- (-30:1) node[below] {$\hat{y}$};
        \draw[line width=2, black, -latex] (0,0) -- (210:1) node[below] {$\hat{x}$};
        \draw[line width=2, black, -latex] (0,0) -- (90:1) node[right] {$\hat{z}$};
    \end{tikzpicture}
\end{center}
    
\section{}

Demostrar las siguientes declaraciones.
    \begin{enumerate}[label=(\alph*)]
        \item El producto vectorial no es asociativo y dados los vectores ${\bf A}$, ${\bf B}$ y ${\bf C}$, se cumple que $$ {\bf A} \times \left( {\bf B} \times {\bf C} \right) = {\bf B} \left( {\bf A} \cdot {\bf C} \right) - {\bf C} \left( {\bf A} \cdot {\bf B} \right). $$
        \item Cualesquiera sean los vectores, se cumple que $$ {\bf A} \times \left( {\bf B} \times {\bf C} \right) + {\bf B} \times \left( {\bf C} \times {\bf A} \right) + {\bf C} \times \left( {\bf A} \times {\bf B} \right) = 0. $$
        \item El producto mixto de tres vectores cualesquiera ${\bf A}$, ${\bf B}$ y ${\bf C}$ es igual al volumen del paralelepípedo construido sobre los mismos una vez llevados a su origen común.
        \item La condición necesaria y suficiente para que tres vectores ${\bf A}$, ${\bf B}$ y ${\bf C}$ sean paralelos a un mismo plano es que su producto mixto sea nulo.
    \end{enumerate}
    
\section{}

Un cuerpo que en el instante $t=0$ se encuentra en un punto A, viaja en línea recta con velocidad constante de módulo desconocido $v$. Cuando transcurre un tiempo $T$, el móvil pasa por un punto B que está a distancia $d$ de A.
    \begin{enumerate}[label=(\alph*)]
        \item Hallar $v$.
        \item Dar dos expresiones para la posición del cuerpo en función del tiempo, una considerando un sistema de coordenadas con origen en A y otra considerando un sistema de coordenadas con origen en B, y graficarlas.
    \end{enumerate}
    
\section{}

Un automóvil viaja en línea recta con velocidad constante desde A hasta C, pasando por B. Se sabe que pasa por A a las 12:00, por B a las 13:00 y por C a las 15:00. (AB = \SI{50}{km}, BC = desconocido).
    \begin{enumerate}[label=(\alph*)]
        \item Elegir un origen de tiempo y un sistema de referencia.
        \item Elegir un instante $t_0$. ¿Cuánto vale $x_0$? Escribir la ecuación de movimiento.
        \item Elegir otro instante $t_0$. ¿Cuánto vale $x_0$? Escribir la ecuación de movimiento.
        \item Calcular la velocidad del auto y la distancia BC.
    \end{enumerate}
    
\section{}
\label{ref:movil}

Un móvil 1 viaja en línea recta desde A hacia B (distancia AB = \SI{300}{km}) a \SI{80}{km/h} y otro móvil 2 lo hace desde B hacia A a \SI{50}{km/h}. El móvil 2 parte 1 hora antes que el móvil 1.
    \begin{enumerate}[label=(\alph*)]
        \item Elegir un origen de tiempo y un sistema de referencia.
        \item Escribir los vectores velocidad ${\bf v}_1$ y ${\bf v}_2$ de los móviles 1 y 2, respectivamente.
        \item En un mismo gráfico, representar la posición en función del tiempo para ambos móviles. ¿Cuál es el significado del punto de intersección de ambas curvas?
        \item En un mismo gráfico, representar la velocidad en función del tiempo para ambos móviles. ¿Cómo podría encontrarse en este gráfico el tiempo de encuentro?
    \end{enumerate}
    
\section{}

Repetir el problema \ref{ref:movil} para el caso en que ambos móviles viajan desde A hacia B.

\section{}

Un cuerpo viaja en línea recta con aceleración constante de módulo desconocido $a$ y dirección como la de la figura. En el instante $t = 0$ el móvil pasa por el punto A con velocidad $v_0$ como la de la figura, en $t = t_0$ el móvil pasa por B y tiene velocidad nula y en $t = t_1$ el móvil pasa por C.
    \begin{enumerate}[label=(\alph*)]
        \item Elegir un sistema de referencia y escribir las expresiones para la posición y la velocidad del móvil en función del tiempo, $x(t)$ y $v(t)$. \label{ref:14a}
        \item Hallar $a$ y la distancia AB.
        \item Calcular la distancia BC y la velocidad del móvil cuando pasa por C, ¿se pueden usar para este cálculo las expresiones $x(t)$ y $v(t)$ del inciso \ref{ref:14a}?
        \item Hallar la velocidad media entre A y B y entre A y C. ¿Coinciden estas dos velocidades medias? ¿Por qué?
    \end{enumerate}
    
\begin{center}
    \begin{tikzpicture}
        \draw[line width=0.8] (0,0) -- (7,0);
        \draw[line width=0.8] (0.5,-0.2) node[below] {C} -- (0.5,0.2);
        \draw[line width=0.8] (2,-0.2) node[below] {A} -- (2,0.2);
        \draw[line width=0.8] (2,0.3) -- (2,0.4);
        \draw[line width=0.8] (6,-0.2) node[below] {B} -- (6,0.2);
        \draw[line width=1.4, -latex] (2,0.3) -- (1,0.3) node[above] {$\bf a$};
        \draw[line width=1.4, -latex] (2,0.3) -- (4,0.3) node[above] {${\bf v}_0$};
        \draw (6,0.3) node[above] {${\bf v} = 0$};
    \end{tikzpicture}
\end{center}

\section{}

Un auto viaja por una ruta a \SI{20}{m/s} cuando un perro se cruza a \SI{50}{m}.
    \begin{enumerate}[label=(\alph*)]
        \item ¿Cómo deben ser los sentidos de los vectores aceleración y velocidad para que el auto frene?
        \item ¿Cuál es la desaceleración mínima que debe imprimirse al automóvil para no chocar al perro? \label{ref:15b}
        \item Idem \ref{ref:15b} teniendo en cuenta que el tiempo de respuesta del chofer es \SI{0,3}{s}. \label{ref:15c}
        \item Mostrar la situación calculada en \ref{ref:15b} y en \ref{ref:15c} en un gráfico de posición en función del tiempo.
    \end{enumerate}

\section{}

Un cuerpo se deja caer desde un globo aerostático que desciende con velocidad \SI{12}{m/s}.
    \begin{enumerate}[label=(\alph*)]
        \item Elegir un sistema de referencia y escribir las ecuaciones que describen el movimiento del cuerpo. \label{ref:16a}
        \item Calcular la velocidad y la distancia recorrida por el cuerpo al cabo de \SI{10}{s}. \label{ref:16b}
        \item Resolver los incisos \ref{ref:16a} y \ref{ref:16b} considerando que el globo asciende a \SI{12}{m/s}.
    \end{enumerate}
    
\section{}

Una piedra en caída libre recorre \SI{67}{m} en el último segundo de su movimiento antes de tocar el piso. Suponiendo que partió del reposo, determinar la altura desde la cual cayó, el tiempo que tarda en llegar al piso y la velocidad de llegada.

\section{}

Desde una terraza a \SI{40}{m} del suelo se lanza hacia arriba una piedra con velocidad \SI{15}{m/s}.
    \begin{enumerate}[label=(\alph*)]
        \item ¿Con qué velocidad vuelve a pasar por el nivel de la terraza?
        \item ¿Cuándo llega al suelo?
        \item ¿Cuándo y dónde se encuentra con una piedra arrojada desde el suelo hacia arriba con una velocidad de \SI{55}{m/s} y que parte desde el suelo en el mismo instante que la anterior?
        \item Representar gráficamente.
    \end{enumerate}

\section{}

Un automóvil cuya velocidad es \SI{90}{km/h} pasa ante un control policial. En ese instante sale en su persecución un patrullero que parte del reposo y acelera uniformemente de modo que alcanza una velocidad de \SI{90}{km/h} en \SI{10}{s}. Hallar:
    \begin{enumerate}[label=(\alph*)]
        \item El tiempo que dura la persecución.
        \item El punto en que el patrullero alcanza el automóvil.
        \item La velocidad del patrullero en el punto de alcance.
    \end{enumerate}

\end{document}
